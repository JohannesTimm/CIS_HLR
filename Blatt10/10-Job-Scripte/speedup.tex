\documentclass[a4paper,12pt]{scrartcl}
\usepackage[ngerman]{babel}
\usepackage[utf8]{inputenc}
\usepackage[verbose]{placeins}
\usepackage{lmodern}
\usepackage{amssymb}
\usepackage{graphicx}
\usepackage{amsmath}
\usepackage{multicol}
\usepackage{enumerate}
\usepackage{verbatim}

\usepackage{float}
%\usepackage{epstopdf}
\usepackage{longtable}
\usepackage{epsfig}
 \usepackage{placeins}
\usepackage{scrpage2}\pagestyle{scrheadings}
\title{Qualitätsprüfung des Parallelisierten PDE-Solvers}
\author{Johannes Timm \and Guannan Hu}
\date{\today}
% \chead{\titleinfo}
% \ohead{\today}
% \setheadsepline{1pt}
% \setcounter{secnumdepth}{0}
% \newcommand{\qed}{\quad \square}

\begin{document}
\maketitle
\notag

\section{Qualitätsprüfung der Implementierung des Gauss-Seidel und Jacobi Iterationsverfahrens}
\subsection{Weak Scaling}
\subsubsection{Gauss-Seidel}
% \begin{multicols}{2}
\begin{figure}[hr!]
 \includegraphics[scale=0.8]{results/WEAK_SCALING_GS.eps}
 \caption{Zusammenhang zwischen Laufzeit, Prozessen, Knoten und Problemgröße für das Gauss-Seidel-Verfahren}
\end{figure}
\begin{table}[hl!]
\verbatiminput{results/WEAK_SCALING_GS.dat}
\caption{Datentabelle zum Weak-Scaling des Gauss-Seidel Verfahrens}
\end{table}
% \end{multicols}


Diesr Graph zeigt die Anzahl der Prozesse, Knoten, Interlines und der 
Laufzeit. Diese Tatsache macht eine Interpretation schwierig. 
Allgemein lässt sich aber beobachten, dass die Schwankungen der Laufzeit, mit Außnahme der Konfiguration (1,1,100), klein sind.
Sie liegen im Bereich von 71-77 Sekunden. Die Überragende Geschwindigkeit bei einem Prozess wird durch das Fehlen der Kommunikation (und dem damit verbundenen Overhead) sowie dem Ausnutzen des CPU-Caches zugeschrieben.
\subsubsection{Jacobi}
% \begin{multicols}{2}
\begin{figure}[hr!]
\includegraphics[scale=0.8]{results/WEAK_SCALING_JA.eps}
 \caption{Zusammenhang zwischen Laufzeit, Prozessen, Knoten und Problemgröße für das Jacobi-Verfahren}
\end{figure}
\begin{table}[hl!]
\verbatiminput{results/WEAK_SCALING_JA.dat}
\caption{Datentabelle zum Weak-Scaling des Jacobi Verfahrens}
\end{table}
% \end{multicols}

Das Jacobi-Verfahren zeigt ein Konträres Verfahren im Vergleich zum Gauss-Seidel-Verfahren. 
Bis auf die letzte Konfiguration liegen die Ausführungszeiten in einer Größenordung zwischen 36 und 40 Sekunden.
Die Letzte Berechnung braucht jedoch 48 Sekunden.
\FloatBarrier
\subsubsection{Wahl der Interlines}
\begin{longtable}{r|c|r|c}
Prozesse&Matrix Größe&Matrix/Prozess&Interlines \\ \hline \endhead 

1	&809	&809	&100\\
2	&1137	&568,5	&141\\
4	&1609	&402,25	&200\\
8	&2257	&282,125	&281\\
16	&3209	&200,5625	&400\\
24	&3929	&163,7083333333	&490\\
64	&6409	&100,140625	&800\\
\caption{Beziehung zwischen Prozessen und Matrixgröße}
\end{longtable}
Die Wahl der Interlines wurde so getroffen, dass die Matrixgröße mit Anzahl der Prozesse zunimmt, aber die Größe der Matrix pro Prozess abnimmt. Die schnellere Berechnung in jedem Prozess sollte dann für das mehr an Kommunikation kompensieren.
Leider sind die Interlines so gewählt, dass eine Lastungleichheit entsteht, da mindestens 1 Prozess eine Zeile mehr oder weniger berechnen braucht als die anderen.
\newpage
\FloatBarrier
\subsection{Strong Scaling}
\subsubsection{Gauss-Seidel}
% \begin{multicols}{2}
\begin{figure}[hr!]
\includegraphics[scale=0.8]{results/STRONG_SCALING_GS.eps}
 \caption{Zusammenhang zwischen Laufzeit, Prozessen und Knoten für das Gauss-Seidel-Verfahren}
\end{figure}
\begin{table}[hl!]
\verbatiminput{results/STRONG_SCALING_GS.dat}
\caption{Datentabelle zum Strong-Scaling des Gauss-Seidel Verfahrens}
\end{table}
% \end{multicols}

Diser Graph zeigt Prozesse, Knoten und benötigte Zeit. Es lässt sich
beobachten, dass der Speedup im Allgemeinen zwar vorhanden ist, aber selbst
mit einer hohen Anzahl an Prozessen schlecht ausfällt.

Auch hier ist eine Interpretation wieder schwierig.

\subsubsection{Jacobi}
% \begin{multicols}{2}
\begin{figure}[hr!]
\includegraphics[scale=0.8]{results/STRONG_SCALING_JA.eps}
 \caption{Zusammenhang zwischen Laufzeit, Prozessen und Knoten für das Jacobi-Verfahren}
\end{figure}
\begin{table}[hl!]
\verbatiminput{results/STRONG_SCALING_JA.dat}
\caption{Datentabelle zum Strong-Scaling des Jacobi-Seidel Verfahrens}
\end{table}
% \end{multicols}

Dieser Graph stellt die gleichen Daten dar wie der vorherige. Es ist zu
beobachten, dass Gauss-Seidel im Vergleich bei einer Hohen Anzahl an Knoten schneller ist als Jacobi.
Zudem profitiert Jacobi offenbar nicht so stark von einer erhöhten Anzahl an Prozessen.
\newpage
\FloatBarrier
\subsection{Communication}
\subsubsection{Gauss-Seidel}
% \begin{multicols}{2}
\begin{figure}[hr!]
\includegraphics[scale=0.8]{results/COMMUNICATION_A_GS.eps}
 \caption{Zusammenhang zwischen Laufzeit und Knoten für das Gauss-Seidel-Verfahren}
\end{figure}
\begin{table}[hl!]
\verbatiminput{results/COMMUNICATION_A_GS.dat}
\caption{Datentabelle zum Skalieren der Kommunikation des Gauss-Seidel Verfahrens}
\end{table}
% \end{multicols}

In diesen Graphen sieht die Beziehung zwischen benötigter Zeit
und Anzahl der Knoten mit insgesamt 10 Prozessen, auf dem das Programm
gleichzeitig ausgeführt wurde. Die benötigte Zeit nimmt tendenziell zu.
Abbruch war nach dem Erreichen einer Genauigkeit von $<3.3504*10^{-5}$.

Bei 4 Knoten lässt sich ein lokales Maximum beobachten, bei 6 Knoten ein lokales Minimum und bei 10 Knoten ein globales Maximum.

Der Anstieg der Zeit lässt sich dadurch erklären, dass bei mehr Nodes der
Aufwand der Kommunkation im Sinne von nötiger (langsamerer) Hardware wächst. Die Kommunikation innerhalb eines Knotens ist wesentlich schneller, als wenn über Knotengrenzen kommuniziert wird.
Dies gilt auch für die Latenz.

% \newpage

\subsubsection{Jacobi}
% \begin{multicols}{2}
\begin{figure}[hr!]
\includegraphics[scale=0.8]{results/COMMUNICATION_A_JA.eps}
 \caption{Zusammenhang zwischen Laufzeit und Knoten für das Jacobi-Verfahren}
\end{figure}
\begin{table}[hl!]
\verbatiminput{results/COMMUNICATION_A_JA.dat}
\caption{Datentabelle zum Skalieren der Kommunikation des Jacobi Verfahrens}
\end{table}
% \end{multicols}

Dieser Graph zeigt die gleichen Parameter wie der vorherige bei gleichen Vorraussetungen, abgesehen davon, dass jetzt das Jacobi Verfahren verwendet wird. 

Das Verhalten ändert sich maßgeblich. Es gibt jetzt keine klar erkennbaren Minima , sondern nur ein globales Maximum bei 10 Knoten. Dieses Maximum ist aber nur halb sogroß wie das vom Gauss-Seidel Verfahren.
Offenbar ist das Jacobi Verfahren nicht ganz so anfällig für Performance Probleme, wie das Gauss-Seidel-Verfahren, wenn es um die Latenz der Kommuikation geht.
\newpage
\FloatBarrier
\subsection{Diskussion}
Es zeigt sich im direkten Vergleich, dass das Gauss-Seidel-Verfahren langsamerer ist als das Jacobi- Verfahren. Dies gilt beim Abbruch nach Genauigkeit. Eigentlich sollte dies nicht so sein, da das Gauss-Seidel-Verfahren mathematisch schneller konvergiert.
Offenbar ist die Implementierung des Gauss-Seidel-Verfahrens nicht genug getunt, bzw. die vorhandene Optimierung ist für die verwendete Rechner-Architektur nicht optimal.
Der Abstand zwischen dem Gauss-Seidel-Verfahren und der hybriden Implementierung des Jacobi-Verfahrens ist erwartbar noch größer. 
Um diese Performance Lücke zu schließen wird empfohlen, dass Gauss-Seidel-Verfahren zu hybridisieren und dies unter Berücksichtigung der NUMA -Architketur des Rechners. Auch das verbessern der Cache-Nutzung scheint empfehlenswert. 
Es ist desweiteren zu prüfen, welche zusätzlichen Compiler-Optionen und Optimierungen hier einen Vorteil bringen. Die Aktivierung von AVX2 und SSE3 Befehlen sollte diesen Programmen Performance Verbesserungen verschaffen. Zur Zeit geht dies jedoch mit Probelmen bei der Nutzung von Debuggern/Memory-Checkern einher. 
Inwieweit die -O3 Optimierungen einen Vorteil bringen und wie sich die Ergebnisse ändern wurde nicht überprüft. 
Als erster Schritt sollten unnötige Instruktionen aus dem Gauss-Seidel Code entfernt werden, da der Compiler diese möglicherweise nicht selbst erkennt.
\end{document}